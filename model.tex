\section{Model for HPC Jobs on BB-Enabled System}
\label{Sec:Model}

%\begin{table}[!t] 
        %\renewcommand{\arraystretch}{1.3}
        %\caption{Summary of Symbols}
        %\label{Tab:Symbols}
        %\centering
        %\begin{tabular}{l|l}
                %\hline
                %$CN$ & number of total compute nodes \\
                %$BB$ & amount of total burst buffer nodes \\
                %$CN_{available}$ & number of available compute nodes \\
                %$BB_{available}$ & amount of available burst buffer nodes \\
                %$c_i$ & compute node demand of $job_i$ \\
                %$bb\_in_i$ & burst buffer demand of $job_i$ at \textit{stage-in} phase \\
                %$bb\_run_i$ & burst buffer demand of $job_i$ at \textit{running} phase \\
                %$bb\_out_i$ & burst buffer demand of $job_i$ at \textit{stage-out} phase \\
                %$rt_i$ & running time of $job_i$ \\
                %$ert_i$ & estimated or expected running time of $job_i$ \\
                %$Q_I, Q_R, Q_O$ & queues for \textit{stage-in, running, stage-out} phase \\
                %$J_{Q_I}, J_{Q_R}, J_{Q_O}$ & set of jobs in corresponding queue \\
                %$S$ & set of jobs selected by scheduler \\
                %$v_i$ & value of $job_i \in J_{Q_R}$ defined by Equation~\ref{Equ:DefValue}\\
                %$dp(\cdot)$ & memo used in dynamic programming \\
                %$TS_{s\_x}$ & event timestamps of starting phase $x$ in simulation \\
                %$TS_{f\_x}$ & event timestamps of finishing phase $x$ in simulation \\
                %\hline
        %\end{tabular}
%\end{table}

\subsection{System Resources}
The system contains resources like compute nodes, main memory, burst buffer,
IO nodes and external storage.
In most system architecture, compute nodes are coupled with main memory.
Since storage capacity at most time is sufficient, we assume IO nodes are always available.
The schedule targets are compute node and burst buffer:
a system with $CN$ compute nodes and $BB$ GB of burst buffer.
BB are able to be shared by applications with any fractional amount.

\subsection{Three-Phase Model for HPC Jobs}
Users request the system to run their applications as jobs.
We define $J = (job_1, job_2, \ldots, job_n)$ to be the set of all unfinished jobs in the system.
To fully utilizing burst buffer, we divide jobs into 3 different phases.
In the first phases, input data are staged in from IO nodes to burst buffer.
We refer this phase as \textit{stage-in} phase.
It is very easy for user to provide the amount of input data
because they should already be available before application execution. 
Then the job can run on the compute nodes after being scheduled;
except reading data from burst buffer, there may be interaction between 
computer node and burst buffer.
These interactions are mainly due to fault tolerance reasons, like check-pointing.
This phase is called \textit{running} phase.
When computation is done, output data needs to be staged out to external PFS from memory,
namely \textit{stage-out} phase.
Burst buffer plays as an efficient IO broker for compute nodes,
in replacing of traditional IO nodes.

\subsection{Three-Phase Resource Demand}
Users typically provide their resource demand for their jobs.
This is the most important information scheduler could get from the users.
Therefore, each job is associated with a demand vector in the form of $(c, rt)$
where $c$ is the number of needed compute node in running phase,
$rt$ is the requested runtime user can provide to help scheduler make decisions.
To be burst buffer aware, demand vector should be augmented
with fields defining user's request for burst buffer capacity.
We consider two possible ways to do so.
Corresponding to the typical usage cases of burst buffer,
user may provide a \textit{burst buffer triple} in the form of $(bb\_in, bb\_run, bb\_out)$,
where $bb\_in$ is the volume of burst buffer user predicted for staging in data files,
$bb\_run$ is the volume of burst buffer user preferred for checkpointing during running,
$bb\_out$ is the volume of burst buffer needed to hold the resulting output data.
A job that providing demand vector augmented with
burst buffer triple is called a \textit{3-phase modeled job}.
Alternatively, a lazy user may just use the $\max\{bb\_in, bb\_run, bb\_out\}$ at every stage.
%We do not make any assumption about $bb\_run$ and $bb\_out$
%because it is nontrivial to predict them, both for system scheduler and application owner.
We refer this augmentation as
\textit{1-phase modeled jobs} thereafter.

The tightly coupling between processing cores and memory makes 3-phase model
more complicated than it appears in terms of resource releasing.
For a job at the stage-in phase, compute nodes is not allocated to it yet
but that will not effect staging in data.
When scheduler allocate compute nodes (coupled with memory) to a job,
these nodes are exclusively used by this job.
The first thing to do in running phase is fetching in data from burst buffer to memory,
which is not available until compute nodes is allocated to job.
We refer this sub-phase as \textit{fectch-in} phase.
$bb\_in$ will be hold by job until \textit{fetch-in} phase finished.
Even if the computation is done when job exit \textit{running phase},
compute nodes cannot be released until job enters \textit{stage-out} phase 
\textbf{and} output data are drained out to burst buffer from memory,
referred as \textit{drain-out} phase.
However, $bb\_run$ can be immediately reclaimed at the end of \textit{running phase}.
Once \textit{drain-out} phase is done,
compute nodes become available for jobs waiting to be run
because staging output data out to IO node is the business of burst buffer nodes,
sometimes with the help from traditional IO nodes.
Burst buffer nodes used for caching output data can thus be put back for reuse
when \textit{stage-out} phase, as well as the entire lifetime of job, concludes.

