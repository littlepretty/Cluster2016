\section{Conclusion and Future Works}
\label{Sec:Conclusion}

In this study we explore batch scheduler allocates burst buffer,
a new storage resource that absorbs 3 general types of IO operations:
data staging in, application checkpointing, and data staging out.
We propose a 3-phase job model tailored to burst buffer's typical usage scenarios.
Buffer aware Cerberus is presented on the basis of the 3-phase job model.
We divide the scheduling problem into 3 sub-phases
and conquer them separately using dynamic programming based optimization.
Simulation results show that
Cerberus could improve both application's and system's performance significantly.


%Since the burst buffer enabled system is not ready for the massive 
%use by the community, 
%there is no study about the burst buffer demand of the typical scientific workloads.
We plan to study how well Cerberus could schedule workloads with diverse resource demands,
which provides valuable insight for computing facilities
configuring their burst buffer subsystem.
Other optimization algorithm can be integrated into Cerberus
to satisfy various scheduling objectives.
We also plan to make Cerberus more versatile
to deal with other new schedulable system resources.

