\section{Conclusion and Future Works}
\label{Sec:Conclusion}

In this study we explore the way HPC workload scheduler allocates burst buffer,
a new storage resources that absorb 3 general types of IO operations:
data staging in, application checkpointing, and data staging out.
We propose a 3-phase job model specially
tailored to burst buffer's typical usage scenarios.
A new burst buffer aware batch scheduler Cerberus is presented
on the basis of the 3-phase job model.
We divide the job scheduling problem into 3 sub-phases
and conquer them separately using dynamic programming based optimization.
Simulation results show that
Cerberus could improve application's perceived IO bandwidth significantly.


Since the burst buffer enabled system is not ready for the massive 
use by the community, 
there is no study about the burst buffer demand of the typical scientific workloads.
We plan to conduct comprehensive sensitivity study about the performance of Cerberus
when scheduling workloads with diversified resource demands. Such study could provide
valuable insight for both computing facilities configure their burst buffer subsystem.
HPC systems users can also benefit from such study by utilizing burst buffer for developing
their scientific applications.
In the future, we also plan to integrate other optimization algorithms into Cerberus
to satisfy various scheduling objectives from the community. 
Cerberus could also become very versatile 
to deal with other new schedulable system resources.

