\section{Conclusion and Future Work}
\label{Sec:Conclusion}

We explore how the batch scheduler can efficiently allocate burst buffer to absorb three general types of IO operations:
data staging in, application checkpointing, and data staging out.
We propose a three-phase job model which is tailored to burst buffer's typical use cases.
Burst-buffer-aware Cerberus is developed on the basis of the three-phase job model.
We divide the scheduling problem into three sub-phases, and conquer them independently using dynamic programming based optimization.
Simulation results show that Cerberus significantly improves both the application-level and the system-level performance.

%Since the burst buffer enabled system is not ready for the massive 
%use by the community, 
%there is no study about the burst buffer demand of the typical scientific workloads.
We plan to study the optimal workload scheduling in Cerberus with diverse resource demands,
which would provide useful insight for computing facilities to configure their burst buffer subsystems.
We will also integrate optimization algorithms to Cerberus to satisfy various scheduling objectives.
We also plan to make Cerberus more versatile to deal with new schedulable system resources.

