\section{Introduction}

%1. Current IO structure
% In today's high-performance computing (HPC) systems,
% application's performance is no longer only throttled by computation capability,
% but also perceived I/O rate between
% numerous amount of parallel processor cores and 
% petabyte volume of storage equipments.
% Current IO architects put IO forwarding nodes in charge of performing IO.
% These IO gateways, together with parallel file system (PFS) software (client side),
% sits between internal system networks that serving communication
% between compute nodes and external system networks
% that interconnects storage nodes\cite{Ross:IOSystem}.
% Applications under this architecture can expect to achieve
% 810 GB/s per PFS sustained bandwidth in Trinity's Lustre\cite{TrinitySystem}
% in later 2016, but still far from the target of 60 TB/s
% for extra-scale computing platform\cite{Shalf:HPCCS:2010}.

% %2. Challenge to current IO architecture and burst buffer come to resuce.
% The challenge roots in the missing gap in HPC's memory hierarchy.
% The ratio of IO rate of memory on the compute node to the storage disk
% is 100 to 10,000 cycles\cite{TrinitySystem}.
% Such a gap makes difference because scientific applications on HPC are exposed to
% bursty IO patterns\cite{Carns:MSST:2011, Kim:PDSW:2010},
% resulting from application's
% defensive IO strategy\cite{Latham:CSD:2012, Naik:ICPPW:2009, Dennis:CUG:2009}
% and the needs of subsequent processing of application output.
% 
% 
% %On one hand, applications checkpoint periodically
% %(so that computation could be restarted after system fault)
% %or store intermediate output for subsequent analysis or visualization;
% %on the other hand, pushing data from memory to external,
% %parallel file system is unproductive due to the IO cycle gap.
% %Though this conflict can be fixed by providing higher IO bandwidth capacity,
% %another character of bursty IO pattern introduces another problem,
% %underutilization of storage system.
% %Production applications could generate hundreds of GB to
% %tens of TB data in one IO request with significant idle interval.
% %For example, observed idle interval of write-intensive jobs
% %reported on Intrepid\cite{Liu:MSST:2012},
% %varies from several minutes to 2 hours.
% 
% %3. Very high level intro to burst buffer
% As an alternative storage design, burst buffer\cite{Bent:HBP:2011, Grider:EXA:2010}
% is targeting on fixing the issues caused by bursty IO pattern.
% It fills the gap in memory hierarchy with storage hardware technology
% faster than traditional disks.
% Bursty IO requests could thus be efficiently absorbed and spread out
% into burst buffer nodes.
% Researchers\cite{Liu:MSST:2012} have demonstrated that application perceived IO
% bandwidth are significantly improved on burst buffer enabled system.
% Given its usefulness, we expect user will explicitly request for
% this new resource upon job submission.


In the field of high performance computing (HPC), 
system performance is no longer throttled only by computation capability,
but also by the ever-increasing I/O gap
between computational resources and disk-based storage technologies.
For example, on the early-state Trinity system~\cite{TrinitySystem}, the I/O nodes between
the compute nodes and the external storage can deliver transfer speed at 810 GB/s,
which is still far from the target of 60 TB/s for the exascale computers\cite{Shalf:HPCCS:2010}.
The gap is critical because scientific applications typically exhibit
``bursty" I/O patterns,
resulting from application's defensive I/O strategy
and the need  for subsequent data processing\cite{Carns:MSST:2011, Kim:PDSW:2010, Latham:CSD:2012, Naik:ICPPW:2009, Dennis:CUG:2009}. 
When multiple applications attempt to access the storage system simultaneously, 
it is easy to saturate
the I/O bandwidth, leading to severe performance degradation.


%3. Very high level intro to burst buffer
Extensive research have been conducted to improve I/O performance on HPC systems from
both hardware and software layers.
We have witnessed a tremendously growing use of flash memory based storage in recent years.
Recently, burst buffer (i.e., a solid state disk based or flash based cache tier)
is proposed to absorb and spread out
the ``bursty" application I/O patterns\cite{Bent:HBP:2011, Grider:EXA:2010}.
% This new cache tier is capable of absorbing and spreading out 
% the ``bursty" application I/O pattern\cite{Bent:HBP:2011, Grider:EXA:2010}.
Studies have demonstrated that the perceived I/O
bandwidth can be significantly improved on burst-buffer-enabled systems\cite{Liu:MSST:2012}.
As such, HPC users are encouraged to explicitly request burst buffer resources at job submission 
on burst-buffer-enabled systems\cite{apex-workflow}.

%4. Our work, a new batch scheduler
In this paper, we propose \textit{Cerberus}\footnote{In Greek mythology,
Cerberus is a monstrous three-headed dog (three-phase scheduler),
who guards the gate of the underworld to the earth (HPC system),
preventing the dead (jobs) from leaving (running).},
a novel batch scheduler for HPC systems equipped with burst buffer. 
We propose a three-phase job model.
%based on the critical use cases of burst buffer, including
%application checkpoint restart and staging in/out data. 
The lifetime of a user job is divided into three phases:
\textit{stage in}, \textit{running}, and \textit{stage-out}.
Burst buffer is used for different purposes in these phases.
Unlike existing batch schedulers that make scheduling decision for each job upon its submission,
Cerberus participates in all of the three phases.
By dividing the lifetime of user jobs into three phases,
Cerberus adopts different optimization strategies for 
making scheduling decision based on job requirement at each phase.
We conduct a series of event driven simulations to evaluate our design. 
As shown later on,
the preliminary results demonstrate that the three-phase design of Cerberus 
not only boosts system responsiveness, but also improves job performance. 
Specifically,
Cerberus accelerates job execution, reduces job response time 
and improves the average system throughput.
Put together, we make three key contributions in this work:
\begin{enumerate}
        \item    We propose a three-phase job model (i.e., stage-in, running, and stage-out) 
        to describe user jobs and further to facilitate job scheduling in a fine granularity.
        
        \item    We present Cerberus, 
        a three-phase burst-buffer-aware scheduler for HPC systems equipped with burst buffer. 
        Our design consists of several key optimization strategies for 
        improving the performance of user jobs throughout different job phases.
        
        \item    We develop an event driven simulator called BBSim 
        for simulating and evaluating Cerberus on the Trinity 
        supercomputer at Los Alamos National Laboratory (LANL)\cite{TrinitySystem}. 
        The simulator is open source and freely available to the community\cite{bbsim-github}.
\end{enumerate}


In the remainder of this paper, Section 2 presents background and 
motivation of using burst buffer on HPC systems.
Section~\ref{Sec:Model} elaborates the three-phase job model.
Section~\ref{Sec:Scheduler} describes the design of Cerberus.
Section~\ref{Sec:Simulation} describes the design of BBsim, the event driven scheduling simulator.
Section~\ref{Sec:Experiments} evaluates Cerberus by comparing it with various alternatives.
Section~\ref{Sec:RelatedWorks} discusses the related works, 
and Section~\ref{Sec:Conclusion} concludes this paper.


